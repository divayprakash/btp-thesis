\documentclass[letterpaper,11pt]{report}

\usepackage{fullpage}
\usepackage{verbatim}
\usepackage{cite}
\usepackage{setspace}
\usepackage{fancyhdr}

\usepackage[small]{caption}
\usepackage{graphics}
\usepackage{color}

\usepackage{hyperref}

\usepackage[dvips]{graphicx}
\graphicspath{ {images/} }

\usepackage{acro}

\setlength{\oddsidemargin}{-0.4mm} % 25 mm left margin
\setlength{\evensidemargin}{\oddsidemargin}
\setlength{\textwidth}{160mm}      % 25 mm right margin
\setlength{\topmargin}{-5.4mm}     % 20 mm top margin
\setlength{\headheight}{5mm}
\setlength{\headsep}{5mm}
\setlength{\footskip}{10mm}
\setlength{\textheight}{237mm}     % 20 mm bottom margin

\setlength{\parskip}{1ex}
\parindent 0in

\def\title{Building Management System : Scheduler and a web service to log data and control devices}

\acsetup{first-style=short}

\DeclareAcronym{ahu}{
  short = AHU ,
  long  = Air Handling Unit ,
  class = abbrev
}
\DeclareAcronym{api}{
  short = API ,
  long  = Application Program Interface ,
  class = abbrev
}
\DeclareAcronym{bms}{
  short = BMS ,
  long  = Building Management System ,
  class = abbrev
}
\DeclareAcronym{css}{
  short = CSS ,
  long  = Cascading Style Sheets ,
  class = abbrev
}
\DeclareAcronym{csu}{
  short = CSU ,
  long  = Ceiling Suspended Unit ,
  class = abbrev
}
\DeclareAcronym{dbms}{
  short = DBMS ,
  long  = Database Management System ,
  class = abbrev
}
\DeclareAcronym{fcu}{
  short = FCU ,
  long  = Fan Coil Unit ,
  class = abbrev
}
\DeclareAcronym{gui}{
  short = GUI ,
  long  = Graphical User Interface ,
  class = abbrev
}
\DeclareAcronym{html}{
  short = HTML ,
  long  = Hypertext Markup Language ,
  class = abbrev
}
\DeclareAcronym{http}{
  short = HTTP ,
  long  = Hypertext Transfer Protocol ,
  class = abbrev
}
\DeclareAcronym{hvac}{
  short = HVAC ,
  long  = Heating Ventilation and Air Conditioning ,
  class = abbrev
}
\DeclareAcronym{io}{
  short = I/O ,
  long  = Input/Output ,
  class = abbrev
}
\DeclareAcronym{lan}{
  short = LAN ,
  long  = Local Area Network ,
  class = abbrev
}
\DeclareAcronym{mac}{
  short = MAC ,
  long  = Media Access Control ,
  class = abbrev
}
\DeclareAcronym{os}{
  short = OS ,
  long  = Operating System ,
  class = abbrev
}
\DeclareAcronym{rest}{
  short = REST ,
  long  = Representational State Transfer ,
  class = abbrev
}
\DeclareAcronym{sd}{
  short = SD ,
  long  = Secure Digital ,
  class = abbrev
}
\DeclareAcronym{soc}{
  short = SoC ,
  long  = System on Chip ,
  class = abbrev
}
\DeclareAcronym{vfd}{
  short = VFD ,
  long  = Variable Frequency Drive ,
  class = abbrev
}
\DeclareAcronym{wifi}{
  short = WiFi ,
  long  = Wireless Fidelity ,
  class = abbrev
}

\begin{document}

\def\degree{B.Tech. in Computer Science and Engineering}
\def\btptrack{Engineering}
\def\submissiondate{April 18, 2017}
\def\supervisor{Prof. Hemant Kumar}
\def\student{Divay Prakash}
\def\rollnumber{2014039}
\def\titlelineone{Building Management System}
\def\titlelinetwo{Scheduler and a web service to log data and control devices}

\thispagestyle{empty}
\vspace{5.65in}

\begin{center}
\vspace{5.65in}
{\LARGE \bf \titlelineone{} : \titlelinetwo{}\\}
\vspace{.3in}
{\Large{Student Name: \student{}}}\\  
{\large{Roll Number: \rollnumber{}}}\\
\vspace{.1in} 
\vspace{.65in}
\vspace{.65in}
{BTP report submitted in partial fulfillment of the requirements 
\\for the Degree of \degree{}\\}
on \submissiondate{}\\
\vspace{.65in}
\textbf{BTP Track}: \btptrack\\
\quad\\
{\textbf{BTP Advisor}\\ 
\supervisor\\} 
\vspace{3.0in}
{Indraprastha Institute of Information Technology\\
New Delhi}
\end{center}


\newpage
\setcounter{page}{2}
\begin{center}
\textbf{\Large Student's Declaration}\label{section:declaration}
\end{center}
We hereby declare that the work presented in the report entitled \textbf{\title{}} submitted by us for the partial fulfillment of the requirements for the degree of \emph{Bachelor of Technology} in \emph{Computer Science \& Engineering} and \emph{Bachelor of Technology} in \emph{Electronics and Communication Engineering} respectively at Indraprastha Institute of Information Technology, Delhi, is an authentic record of our work carried out under guidance of \textbf{Prof. Hemant Kumar}. Due acknowledgements have  been given in the report to all material used. This work has not been submitted anywhere else for the reward of any other degree.\\
\vspace{0.5in}

\textbf{..............................}\hfill
\textbf{ Place \& Date: .............................}\\
\textbf{Divay Prakash, Amogh Vithalkar}

\vspace{3in}
\begin{center}
\textbf{\Large Certificate} \label{section:certificate}
\end{center}
This is to certify that the above statement made by the candidate is correct to the best of my knowledge.\\
\vspace{0.4in}

\textbf{..............................}\hfill
\textbf{ Place \& Date: .............................}\\
\textbf{Prof. Hemant Kumar}

\pagebreak

\begin{abstract}
The building management HVAC(heating, ventilation and air conditioning) system for Phase II of IIIT-Delhi campus is designed to take care (and advantage) of diversity of use. Instead of using large AHUs(Air Handling Units), we will have individual units in faculty rooms, labs, and other spaces. This would allow us to condition air of the spaces that are occupied and the system would be able to maintain desired temperature more closely. The disadvantage of this approach is higher capital cost and a larger I/O points for BMS but the running cost will be saved. In this project we are developing a scheduler hosted on a web server to control the AC’s valves connected through Raspberry Pi and Arduino.
\par
\vspace{2.15in}
Keywords: building management system, scheduler, web server 
\end{abstract}

\newpage
\section*{Acknowledgments}\label{section:acknowledgments}
\pagestyle{plain}
\pagenumbering{roman}
We take this opportunity to express our deepest gratitude and appreciation to all those
who have helped us directly or indirectly towards the successful completion of this project.
\par
We would like to express our sincere gratitude to our advisor Prof. Hemant Kumar for providing his invaluable guidance, comments and suggestions throughout the course of the project.

\vspace{2in}

\section*{Work Distribution}\label{section:workdistribution}
The distribution of work done by the team members over the course of this project is as\\ follows -
\begin{itemize}
    \item Divay Prakash
    \begin{itemize}
        \item Designed and implemented the frontend for the central server
        \item Documented Python code written for the sub-server
        \item Documented Arduino code written for the microcontroller
        \item Ported sub-server from \verb|BaseHTTPServer| to Django server
    \end{itemize}
    \item Amogh Vithalkar
    \begin{itemize}
        \item Implemented the backend for the central server
        \item Built a scheduler for the central server
    \end{itemize}
\end{itemize}

\newpage
\tableofcontents

\listoffigures
\addcontentsline{toc}{chapter}{List of Figures}

\newpage
\addcontentsline{toc}{chapter}{List of Abbreviations}
\chapter*{List of Abbreviations}
\printacronyms[include-classes=abbrev, heading=none]

\newpage
\chapter{Introduction}\label{chapter:introduction}
\pagenumbering{arabic}
\setcounter{page}{1}
\onehalfspacing
\section{BMS for phase I}
\ac{hvac} for the Phase I buildings (Academic, Lecture halls and library building) of IIIT-Delhi campus used large floor mounted \ac{ahu}'s to condition large spaces except lecture halls which were served by \ac{csu}'s (ceiling suspended units). Library building has one \ac{ahu} per floor feeding all the labs and rooms. The academic building has 6 AHUs (three serving A wing and the other three wing B). Hostels have one \ac{fcu} per room. The large AHUs of the academic block control volume of cold air by reducing the fan speed using \ac{vfd} drives. As the area per \ac{ahu} is large and the need very diverse (labs with varying number of occupants and equipment vs faculty rooms), the \ac{hvac} system is not very energy efficient and uniform temperature can't be maintained. The \ac{hvac} system has a very basic centralized control that can set temperature to be achieved in each one of the \ac{ahu}s or \ac{csu}s. It also allows switching on/ off, monitoring parameters etc centrally. Other than \ac{ahu}s/\ac{csu}s it can display status and parameters of chillers, cooling towers and hot water generator.
\par
In \ac{bms}/ \ac{hvac} terminology each point that is controlled or read is an \ac{io} point and an \ac{io} summary is prepared for any BMS installation to estimate the cost. The \ac{io} points are of 4 types - digital input, digital output, analog input and analog output. In phase I, we had chosen to only include \ac{io} that either needed to be controlled or the \ac{io} that were to be sensed and used for controlling to contain cost. The system was provided by Trane.
\par
Later a parallel system for monitoring and data collection was implemented by a research group led by Dr Amarjeet Singh. This group installed wireless temperature sensors (13) and Ethernet based power meters (500) all over the campus to optimize HVAC and energy use based on data collected.
\pagebreak
\section{\ac{bms} for phase II}
The \ac{hvac} for Phase II of IIIT-Delhi campus is designed to take care (and advantage) of diversity of use. So instead of large \ac{ahu}s, we will have individual units in faculty rooms, labs, and other spaces. This would allow us to condition air of the spaces that are occupied and the system would be able to maintain desired temperature more closely. The disadvantage of this approach is higher capital cost and a larger \ac{io} points for \ac{bms} but the running cost will be saved.

\newpage
\chapter{Problem Statement}\label{chapter:problemstatement}
\onehalfspacing
Given the large number of units to monitor and control, the cost of \ac{bms} was estimated to be very high. In addition, for phase II, the traditional system would be close-ended and proprietary.
\par
Most of the cost (about 60\%) would have been in the wiring the large number of units to controllers. This cost can be saved if individual \ac{wifi}-based controllers (500-600) were to be deployed. These wireless devices can use the existing institute \ac{lan} infrastructure to further save costs.

\newpage
\chapter{Architecture}\label{chapter:architecture}
\onehalfspacing
\begin{figure}[h]
\includegraphics[width=6cm, height=10cm]{arch}
\centering
\captionsetup{justification=centering}
\caption{Block diagram of the building management system project}
\label{fig:arch}
\end{figure}
The \ac{bms}/ \ac{hvac} system, is structured in the manner described by figure \ref{fig:arch}. At the lowest level is the AHU (Air Handling Unit), which performs the actual cooling/heating functionality of the system. It is controlled by the microcontroller. The microcontroller monitors various system parameters and accordingly runs the \ac{ahu}. It is in turn controlled by a sub-server unit, which is responsible for logging data and passing control instructions to the microcontroller unit according to policies set by the central server. The highlighted sections in figure \ref{fig:arch} are the modules that were worked upon over the course of this project. In addition, code documentation for the entire stack was also written.

\newpage
\chapter{Design}\label{chapter:Design}
\onehalfspacing
\section{Overview}
\begin{figure}[h]
\includegraphics[width=4cm, height=4.5cm]{des}
\centering
\captionsetup{justification=centering}
\caption{Higher level control modules of the building management system}
\label{fig:des}
\end{figure}
The higher level control of the building management system is carried out by the central server in conjunction with various sub-server units managing their own group of microcontroller devices. The details of the central server and the sub-servers are given in the following sections.
\section{Central server}
The central server of the system is a software implementation only. There are no specific hardware requirements for the same. The server is implemented in such a manner so as to be able to provide interfaces to both the sub-server devices using a \ac{rest} \ac{api} and also to system administrators by way of a \ac{gui}. If required, the central server could be on the cloud. The server can be broken down into two main modules, the frontend and the backend. Both of these modules were built over the course of this project.
\subsection{Design specifications}
\subsubsection{Frontend}
The frontend of the central server consists of a hierarchy of web pages created using \ac{html}/\newline \ac{css}/JavaScript/jQuery and the Bootstrap framework. This provides the \ac{gui} which can be used by system administrators for overall analysis/control of the system.
\subsubsection{Backend}
The backend of the central server consists of a web application created using the Python web framework Django. The system used MySQL as the database management system.
\subsection{Design choices}
There are various design choices that had to be made in order to create a cohesive and well-structured central server for the \ac{bms} project. The central server is implemented as a web application. The choice of a web-application was made keeping in mind the advantages of web applications over traditional desktop applications -
\begin{itemize}
    \item Easier deployment - The application does not need to be individual deployed to any client machine. The client machine only requires a functioning web browser to access the system.
    \item Maintenance - System maintenance is vastly easier as all updates and bug fixes need to be introduced at the server end only. In the same manner, regular updates are also easy.
    \item Platform independent - This was the major reason in favour of using a web application. The server implemented as a web application provides complete functionality to clients running any \ac{os}.
    \item Faster development process - By using a web application, users access the system by way of a web browser. This creates a uniform environment across platforms. While user interaction with the application needs to be thoroughly tested on different web browsers, the application itself needs only be developed for a single operating system. There’s no need to develop and test it on all possible operating system versions and configurations. This makes development and troubleshooting much easier.
    \item Scalability - Being independent of hardware configurations, web applications are easily scalable. It is possible to scale the system with growing \ac{io} points or number of client instances.
\end{itemize}
\section{Sub-server}
\subsection{Design specifications}
The sub-servers are individual Raspberry Pi units running a Linux \ac{os} distribution. These devices are connected to the central server using a wired Ethernet connection and each unit manages up to 20 AVR microcontroller units over a \ac{wifi} interface. The Raspberry Pi devices host a web server which provides various functionalities. Over the course of this project, the web server was ported from a \verb|BaseHTTPServer| to a Django-based server, both coded in Python.
\subsection{Design choices}
The choice of using a Raspberry Pi device for the sub-server was made considering the following points -
\begin{itemize}
    \item Ease of deployment - Due to the small size of the device, it is easy to deploy unobtrusively in extremely small spaces such as wiring closets and channels, where a power supply and Ethernet cables are available.
    \item \ac{soc} - The Raspberry Pi provides us a complete system on chip ie. it integrates all the components of a computer on one single chip. This micro-computer does not require any additional chips for its functionality, with built-in Ethernet and \ac{wifi} interfaces.
\end{itemize}

\newpage
\chapter{Implementation}\label{chapter:Implementation}
\onehalfspacing
\section{Central server}
\subsection{Description}
The central server is a Django web application which uses a MySQL \ac{dbms}. The frontend part of the web application is done using \ac{html}/\ac{css} and bootstrap.
\subsection{Functionality}
The web application takes inputs from the user to set the time and temperature range for each micro-controller for which the \ac{ahu}'s should work and logs the data in the MySQL database on the server according to the microcontroller's \ac{mac} address. Central server communicates to sub-servers using the client-server model.
\section{Sub-server}
\subsection{Description}
The Raspberry Pi units serving as sub-servers run a Django based web application which serves a \ac{rest} \ac{api}. This is utilised by both the microcontrollers and the central server which communicate with the device using \ac{http} messages. In addition, the sub-server device also functions as a client in some cases. This is further explained in the next section.
\subsection{Functionality}
\subsubsection{For microcontroller}
All communication taking place between sub-server devices and the microcontrollers follows the client-server model. However, the server (Rapberry Pi device) cannot initiate a message send to the client (microcontroller) without a prior request from the client. Due to memory and threading constraints at the microcontroller end, interrupts have not been used. Thus all communication is initiated by the client device. 
\begin{itemize}
    \item Data logging - The sub-server device provides a \ac{rest} \ac{api} method for microcontroller devices to log data. The microcontroller devices makes an \ac{http} POST request to the Raspberry Pi sub-server, which processes the attached data and stores it in the database. This data is stored in a MySQL database using microcontroller \ac{mac} addresses and the \ac{http} message timestamp as keys. Thereafter, the sub-server sends a confirmation message back to the client.
    \item Fetching commands - The microcontroller devices also fetch commands from the sub-server units. For this purpose, the microcontroller makes an \ac{http} GET request to the sub-server. The sub-server extracts the microcontroller's \ac{mac} address from the request and queries its database for any pending commands to be sent to that device. If found, it returns the same in the \ac{http} response to the client, else an empty response is sent.
\end{itemize}
\subsubsection{For central server}
In case of the communication with the central server, the model followed is again client-server model. However, both the sub-server and central server can initiate message sending.
\begin{itemize}
    \item Receive commands - The sub-server device acts as a client for this method, used by the central server to transmit commands to the sub-server using \ac{http} messages.
    \item Send data
    \begin{itemize}
        \item With prior request - This method is followed if the central server requests data logs from the sub-server using an \ac{http} GET message as-and-when required.
        \item With no prior request - The Raspberry Pi device uses an \ac{sd} card to store data persistently. To ensure longevity of the system, it is essential to keep the number of read/write cycles on the card to a minimum. Thus data logs are stored in primary memory. While this solves the \ac{sd} card issue, it creates another as main system memory is being taken up by static data. To resolve this issue, the Raspberry Pi device dumps the data logs to the central server using \ac{http} POST messages at a fixed time interval. This enables the deletion of the files at sub-server end, freeing up valuable memory resources.
    \end{itemize}
\end{itemize}

\newpage
\chapter{Specification}\label{chapter:Specification}
\onehalfspacing
\begin{figure}[h]
\includegraphics[width=\textwidth]{index}
\centering
\captionsetup{justification=centering}
\caption{Home page}
\label{fig:index}
\end{figure}
The web application starts with this page. The user has to login to the application to use its features. Once the user logs in to the application, they can control individual sub-servers and microcontrollers. The user can also see the current system statistics as well as the previous parameter settings.
\par
If the user wants to manage a sub-server, the user can navigate through the Raspberry Pi section to reach the following page (figure \ref{fig:list}).

\newpage
\begin{figure}[h]
\includegraphics[width=\textwidth]{list}
\centering
\captionsetup{justification=centering}
\caption{List of available sub-servers}
\label{fig:list}
\end{figure}
In this page the user selects the MAC address of the sub-server unit and is redirected to the following page (figure \ref{fig:raspberrypi}). This page also shows the location and status of the sub-server units.

\newpage
\begin{figure}[h]
\includegraphics[width=\textwidth]{raspberrypi}
\centering
\captionsetup{justification=centering}
\caption{Managing a sub-server}
\label{fig:raspberrypi}
\end{figure}
This page displays the information of the sub-server. The user can set the global policy for all the microcontrollers connected to the sub-server. The global policy includes the temperature range and time range in which the AHU’s should operate. The user can also view the past records of the sub-server. Once the user clicks submit, the data is entered in MySQL database. If the user wants to manage individual microcontroller, the user can navigate through the Arduino section in the home page to reach the following page (figure \ref{fig:block}).

\newpage
\begin{figure}[h]
\includegraphics[width=\textwidth]{block}
\centering
\captionsetup{justification=centering}
\caption{Blocks in Phase II}
\label{fig:block}
\end{figure}
The sub-servers in phase II are distributed in various blocks. In this page the user has to select the block for which the microcontroller needs to be managed. On selecting the block, the user will be directed to the following page (figure \ref{fig:wing}).

\newpage
\begin{figure}[h]
\includegraphics[width=\textwidth]{wing}
\centering
\captionsetup{justification=centering}
\caption{Wings in a block}
\label{fig:wing}
\end{figure}
Depending on the block, the sub-servers are then divided into wings. In this page the user has to select the wing for which the user wants to manage the device. The user will then be redirected to the following page (figure \ref{fig:floor}).

\newpage
\begin{figure}[h]
\includegraphics[width=\textwidth]{floor}
\centering
\captionsetup{justification=centering}
\caption{Floors in a wing}
\label{fig:floor}
\end{figure}
Depending on the wing, the sub-servers are further divided into different floors. In this page the user has to select the floor for which the microcontroller needs to be managed. On selecting the floor, the user will be directed to the following page (figure \ref{fig:room}).

\newpage
\begin{figure}[h]
\includegraphics[width=\textwidth]{room}
\centering
\captionsetup{justification=centering}
\caption{Rooms in a floor}
\label{fig:room}
\end{figure}
In this page the user has to select the room for which the user wants to manage the microcontroller unit. The user will then be redirected to the following page (figure \ref{fig:arduino}).

\newpage
\begin{figure}[h]
\includegraphics[width=\textwidth]{arduino}
\centering
\captionsetup{justification=centering}
\caption{Managing the microcontroller}
\label{fig:arduino}
\end{figure}
This page displays the information of the microcontroller. The user can set the policy for the microcontroller. The policy includes the temperature range and time range in which the AHU should operate. The user can also view the past records of the microcontroller. Once the user clicks submit, the data is entered in MySQL database.
\par
The central server then communicates with the python server to pass the command. A Python based task scheduler schedules the task for the sub-server and microcontroller according to the values in the database.

\newpage
\nocite*{}
\bibliographystyle{acm}
\addcontentsline{toc}{chapter}{Bibliography}
\bibliography{bibdb}

\end{document}
